%%%%%%%%%%%%%%%%%%%%%%%%%%%%%%%%%%%%%%%%%
% Programming/Coding Assignment
% LaTeX Template
%
% This template has been downloaded from:
% http://www.latextemplates.com
%
% Original author:
% Ted Pavlic (http://www.tedpavlic.com)
%
% Note:
% The \lipsum[#] commands throughout this template generate dummy text
% to fill the template out. These commands should all be removed when
% writing assignment content.
%
% This template uses a Perl script as an example snippet of code, most other
% languages are also usable. Configure them in the "CODE INCLUSION
% CONFIGURATION" section.
%
%%%%%%%%%%%%%%%%%%%%%%%%%%%%%%%%%%%%%%%%%

%----------------------------------------------------------------------------------------
%	PACKAGES AND OTHER DOCUMENT CONFIGURATIONS
%----------------------------------------------------------------------------------------

\documentclass{article}

\usepackage{fancyhdr} % Required for custom headers
\usepackage{lastpage} % Required to determine the last page for the footer
\usepackage{extramarks} % Required for headers and footers
\usepackage[usenames,dvipsnames]{color} % Required for custom colors
\usepackage{graphicx} % Required to insert images
\usepackage{listings} % Required for insertion of code
\usepackage{courier} % Required for the courier font
\usepackage{lipsum} % Used for inserting dummy 'Lorem ipsum' text into the template

% Margins
\topmargin=-0.45in
\evensidemargin=0in
\oddsidemargin=0in
\textwidth=6.5in
\textheight=9.0in
\headsep=0.25in

\linespread{1.1} % Line spacing

% Set up the header and footer
\pagestyle{fancy}
\lhead{\hmwkAuthorName} % Top left header
\chead{\hmwkClass\ : \hmwkTitle} % Top center head
\rhead{\firstxmark} % Top right header
\lfoot{\lastxmark} % Bottom left footer
\cfoot{} % Bottom center footer
\rfoot{Page\ \thepage\ of\ \protect\pageref{LastPage}} % Bottom right footer
\renewcommand\headrulewidth{0.4pt} % Size of the header rule
\renewcommand\footrulewidth{0.4pt} % Size of the footer rule

\setlength\parindent{0pt} % Removes all indentation from paragraphs

%----------------------------------------------------------------------------------------
%	CODE INCLUSION CONFIGURATION
%----------------------------------------------------------------------------------------

\definecolor{MyDarkGreen}{rgb}{0.0,0.4,0.0} % This is the color used for comments
\lstloadlanguages{Perl} % Load Perl syntax for listings, for a list of other languages supported see: ftp://ftp.tex.ac.uk/tex-archive/macros/latex/contrib/listings/listings.pdf
\lstset{language=Perl, % Use Perl in this example
        frame=single, % Single frame around code
        basicstyle=\small\ttfamily, % Use small true type font
        keywordstyle=[1]\color{Blue}\bf, % Perl functions bold and blue
        keywordstyle=[2]\color{Purple}, % Perl function arguments purple
        keywordstyle=[3]\color{Blue}\underbar, % Custom functions underlined and blue
        identifierstyle=, % Nothing special about identifiers
        commentstyle=\usefont{T1}{pcr}{m}{sl}\color{MyDarkGreen}\small, % Comments small dark green courier font
        stringstyle=\color{Purple}, % Strings are purple
        showstringspaces=false, % Don't put marks in string spaces
        tabsize=5, % 5 spaces per tab
        %
        % Put standard Perl functions not included in the default language here
        morekeywords={rand},
        %
        % Put Perl function parameters here
        morekeywords=[2]{on, off, interp},
        %
        % Put user defined functions here
        morekeywords=[3]{test},
       	%
        morecomment=[l][\color{Blue}]{...}, % Line continuation (...) like blue comment
        numbers=left, % Line numbers on left
        firstnumber=1, % Line numbers start with line 1
        numberstyle=\tiny\color{Blue}, % Line numbers are blue and small
        stepnumber=5 % Line numbers go in steps of 5
}

% Creates a new command to include a perl script, the first parameter is the filename of the script (without .pl), the second parameter is the caption
\newcommand{\perlscript}[2]{
\begin{itemize}
\item[]\lstinputlisting[caption=#2,label=#1]{#1.pl}
\end{itemize}
}

%----------------------------------------------------------------------------------------
%	DOCUMENT STRUCTURE COMMANDS
%	Skip this unless you know what you're doing
%----------------------------------------------------------------------------------------

% Header and footer for when a page split occurs within a problem environment
\newcommand{\enterProblemHeader}[1]{
\nobreak\extramarks{#1}{#1 continued on next page\ldots}\nobreak
\nobreak\extramarks{#1 (continued)}{#1 continued on next page\ldots}\nobreak
}

% Header and footer for when a page split occurs between problem environments
\newcommand{\exitProblemHeader}[1]{
\nobreak\extramarks{#1 (continued)}{#1 continued on next page\ldots}\nobreak
\nobreak\extramarks{#1}{}\nobreak
}

\setcounter{secnumdepth}{0} % Removes default section numbers
\newcounter{homeworkProblemCounter} % Creates a counter to keep track of the number of problems

\newcommand{\homeworkProblemName}{}
\newenvironment{homeworkProblem}[1][Problem \arabic{homeworkProblemCounter}]{ % Makes a new environment called homeworkProblem which takes 1 argument (custom name) but the default is "Problem #"
\stepcounter{homeworkProblemCounter} % Increase counter for number of problems
\renewcommand{\homeworkProblemName}{#1} % Assign \homeworkProblemName the name of the problem
\section{\homeworkProblemName} % Make a section in the document with the custom problem count
\enterProblemHeader{\homeworkProblemName} % Header and footer within the environment
}{
\exitProblemHeader{\homeworkProblemName} % Header and footer after the environment
}

\newcommand{\problemAnswer}[1]{ % Defines the problem answer command with the content as the only argument
\noindent\framebox[\columnwidth][c]{\begin{minipage}{0.98\columnwidth}#1\end{minipage}} % Makes the box around the problem answer and puts the content inside
}

\newcommand{\homeworkSectionName}{}
\newenvironment{homeworkSection}[1]{ % New environment for sections within homework problems, takes 1 argument - the name of the section
\renewcommand{\homeworkSectionName}{#1} % Assign \homeworkSectionName to the name of the section from the environment argument
\subsection{\homeworkSectionName} % Make a subsection with the custom name of the subsection
\enterProblemHeader{\homeworkProblemName\ [\homeworkSectionName]} % Header and footer within the environment
}{
\enterProblemHeader{\homeworkProblemName} % Header and footer after the environment
}

%----------------------------------------------------------------------------------------
%	NAME AND CLASS SECTION
%----------------------------------------------------------------------------------------

\newcommand{\hmwkTitle}{Assignment\ \#1} % Assignment title
\newcommand{\hmwkDueDate}{Friday,\ Feb\ 5th,\ 2016} % Due date
\newcommand{\hmwkClass}{Design of Optimizing Compilers\ CS5214} % Course/class
\newcommand{\hmwkClassTime}{18:30} % Class/lecture time
\newcommand{\hmwkClassInstructor}{Weng-Fai Wong} % Teacher/lecturer
\newcommand{\hmwkAuthorName}{Pan An(A0134556A)} % Your name

%----------------------------------------------------------------------------------------
%	TITLE PAGE
%----------------------------------------------------------------------------------------

\title{
\vspace{2in}
\textmd{\textbf{\hmwkClass:\ \hmwkTitle}}\\
\normalsize\vspace{0.1in}\small{Due\ on\ \hmwkDueDate}\\
\vspace{0.1in}\large{\textit{\hmwkClassInstructor\ \hmwkClassTime}}
\vspace{3in}
}

\author{\textbf{\hmwkAuthorName}}
\date{} % Insert date here if you want it to appear below your name

%----------------------------------------------------------------------------------------

\begin{document}

\maketitle

%----------------------------------------------------------------------------------------
%	TABLE OF CONTENTS
%----------------------------------------------------------------------------------------

%\setcounter{tocdepth}{1} % Uncomment this line if you don't want subsections listed in the ToC

\newpage
\tableofcontents
\newpage

%------------------------------------------
%\begin{homeworkSection}{}
%
%\end{homeworkSection}
%----------------------------------------------
%	PROBLEM 1
%----------------------------------------------------------------------------------------

% To have just one problem per page, simply put a \clearpage after each problem

\begin{homeworkSection}{Design of Calculator}
This calculator is designed and implemented in Mac OS X system, any
unix compatible system can be able to install and run the program. The
calculator is  designed with the utilization of FLex and Bison. Basic
functionalities in the original design is listed below:

\begin{itemize}
\item Numerical system. The calculator should be able to detect
  different data types including: integers, floating point numbers and
  hexadecimals.
\item Operations. The Calculator should be able to conduct basic
  BODMAS rules. Also it should support basic scientific calculations
  like trigonometric and logrithmic operations.
\item Batch processing. The calculator should be able to batch process
  many calculations from a specific file.
\item Symbolic computation. This feature is not implemented due to the
  lack of time. But the calculator should be able to integrate with
  symbolic engines and conduct complex mathmatical operations.
\end{itemize}

The detailed design and introduction about the calculator is listed in
the following sections. The tarball name of my project is:
$$Pan An‐assignment‐1.tar.gz$$


\end{homeworkSection}

%----------------------------------------------------------------------------------------
%	PROBLEM 2
%----------------------------------------------------------------------------------------
\begin{homeworkSection}{Installation}
The installation of the calculator can be easily done with a
customized {\bf makefile}. Open terminal and switch to the calculator
folder:

\vspace{2mm}
{\bf
$\$ make   $

$\$ ./calculator $
}

\vspace{2mm}

The calculator should be running. You can type in any tasks, if the
task expression is wrong, the calculator will report error and exit.

You can also put multiple tasks to {\bf test.txt}, then enter the
calculator and type:

\vspace{2mm}

{\bf
$\$ test   $
}

\vspace{2mm}


the results of all the tasks(before the first wrong task expression)
should be processed and displayed.

\end{homeworkSection}



\begin{homeworkSection}{Numbering System and Basic Calculation}
The numbering system of the calculator supports basic integer,
scientific floating number and hexadecimals.

Noticing that hexadecimals in this case is all considered and will be
transformed to 32-bit integers in any task.

Some correct case of scientific floating expression is listed below:
\begin{itemize}
\item .1 (0.1)
\item 1e3 (1000)
\item 1e-2 (0.01)
\item .1e (0.1)
\item -0.1e-4 (-0.00001)
\end{itemize}

The calcolator supports BODMAS rules expressions. Basic operations
includes: +, -, *, / and ** (power).

Here is an example of task that contains different operations:

$$(2+3**3)**(3*(4+6)-12)$$

and the result is:
$210457279823473787946401792.000000$.

\end{homeworkSection}


\begin{homeworkSection}{Scientific Calculation}


Supported scientific operations are listed below:
\begin{itemize}
\item Constant number pi
\item Trigonometric and hyperbolic: sin, cos, tan, sinh, cosh, tanh
\item Standard functions: mod, ceil, floor, abs, sqrt
\item Logrithmics: log2 and log10
\end{itemize}
Here is an example of task that contains different operations:

$$(sin(.5*pi)) + (cos(pi))**2$$

Noticing that all of the scientific operation expressions should have
extra '()' around it.


\end{homeworkSection}

\begin{homeworkSection}{Challenges and Solutions}
During implementation of this calculator, one of the challanges is
that when designing the calculator the designer should be completely
clear about each data types and operations. For example: floating
numbers it self have a lot of cases that should be taken into
consideration. By using better designed regular expression I was able
to make sure that it does not go wrong. By splitting up integer number
and floating number into different expression definition block, I was
able to manipulate the whole calculation structure without going
wrong.

Also the utilization of third class libraries greatly reduced the
amount of work done my myself.

Flex itself controls the whole process of the script, one of the main
difficulties to make the calculator better accepted by everybody. The
parsing and loop controlling is controlled by FLex itself. The only
way to make this work is to override different functions. I used some
function pointer tricks to make it look better when the program goes
wrong. Like when the calculator finished the test.txt batch processing
the program is supposed to give an error saying invalid syntax because
$EOF$ is not gonna be processed correctly. I override the flex
function and pointed out come to it, so what you get is:
$$All Results from Test File!!$$

One thing that I did not achieve in my original design is the symbolic
engine integration. Since Python has provided strong and mature
packages for symbolic calculation, I was able to model a better
version of calculator by having some more expression definitions and
having the calculator interact with python scripts. However the time
is not enough for me to implement all these features.

Another difficulties is to consider the operating system compatibility
of the calculator. Some of flex libraries have different compilation
rules in different operating systems. Luckily I solved with a simple
Makefile.



\end{homeworkSection}


\end{document}
%%% Local Variables:
%%% mode: latex
%%% TeX-master: t
%%% End:
