% Created 2016-04-14 Thu 21:52
\documentclass[11pt]{article}
\usepackage[utf8]{inputenc}
\usepackage[T1]{fontenc}
\usepackage{fixltx2e}
\usepackage{graphicx}
\usepackage{grffile}
\usepackage{longtable}
\usepackage{wrapfig}
\usepackage{rotating}
\usepackage[normalem]{ulem}
\usepackage{amsmath}
\usepackage{textcomp}
\usepackage{amssymb}
\usepackage{capt-of}
\usepackage{hyperref}
\author{潘安}
\date{\today}
\title{}
\hypersetup{
 pdfauthor={潘安},
 pdftitle={},
 pdfkeywords={},
 pdfsubject={},
 pdfcreator={Emacs 24.5.1 (Org mode 8.3.4)}, 
 pdflang={English}}
\begin{document}

\tableofcontents

Org-mode Testing Report
\section{Matrix Multiplication with GCC}
\label{sec:orgheadline3}

\subsection{Abstract}
\label{sec:orgheadline1}
Large matrix multiplications can be very handy considering 
the limitations of cache and the speed of the the RAM access.
This report is about conducting matrix multiplication with 
low level optimizations. Loop tiling and LLVM optimizations
are used in order to provide an acceptable result for a better
scheduling and work load reduction.

\subsection{Basic Matrix Multiplication}
\label{sec:orgheadline2}
The first version of matrix multiplication is done with
 two arrays of integers(simply for testing). My computer is 
an Alienware 17 R5, and some of the hardware specifications are listed
below:
\begin{center}
\begin{tabular}{lr}
Params & Value\\
\hline
Architecture: & x86\(_{\text{64}}\)\\
CPU op-mode(s): & 32-bit, 64-bit\\
Byte Order: & Little Endian\\
CPU(s): & 8\\
On-line CPU(s) list: & 0-7\\
Thread(s) per core: & 2\\
Core(s) per socket: & 4\\
Socket(s): & 1\\
NUMA node(s): & 1\\
Vendor ID: & GenuineIntel\\
CPU family: & 6\\
Model: & 60\\
Stepping: & 3\\
CPU MHz: & 901.875\\
BogoMIPS: & 4788.98\\
Virtualization: & VT-x\\
L1d cache: & 32K\\
L1i cache: & 32K\\
L2 cache: & 256K\\
L3 cache: & 6144K\\
NUMA node0 CPU(s): & 0-7\\
\end{tabular}
\end{center}
\end{document}
