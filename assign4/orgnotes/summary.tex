% Created 2016-04-27 Wed 03:48
\documentclass[11pt]{article}
\usepackage[utf8]{inputenc}
\usepackage[T1]{fontenc}
\usepackage{fixltx2e}
\usepackage{graphicx}
\usepackage{grffile}
\usepackage{longtable}
\usepackage{wrapfig}
\usepackage{rotating}
\usepackage[normalem]{ulem}
\usepackage{amsmath}
\usepackage{textcomp}
\usepackage{amssymb}
\usepackage{capt-of}
\usepackage{hyperref}
\author{Pan An (A0134556A)}
\date{\today}
\title{Paper Presentation Notes}
\hypersetup{
 pdfauthor={Pan An (A0134556A)},
 pdftitle={Paper Presentation Notes},
 pdfkeywords={},
 pdfsubject={},
 pdfcreator={Emacs 24.5.1 (Org mode 8.3.4)}, 
 pdflang={English}}
\begin{document}

\maketitle
\tableofcontents

\section{OPGEN}
\label{sec:orgheadline1}
Every compiler has its optimizing rules. These rules, most of the time, are made by compiler developers with a good mathematical background. 
Hand crafted rules, however, can never guarantee correctness and completeness. 

OPGEN solves this problem by generating all the local optimizations. Optgen generates optimization rules that work on the IR level. 
This allows a more abstract view on the program behavior than working on assembly level. The basic idea
of OPGEN is to generate all instruction blocks and compares the output. If the generated expressions pass all the test cases AND
at the same time has a smaller cost, these expressions will be selected. 

Testing of OPGEN on modern compilers has helped identifying more than 50 missing optimizations.
\section{Exploring and Enforcing Security Guarantees via Program Dependence Graphs}
\label{sec:orgheadline2}

PIDIN is a program analysis and understanding tool that enables the specification and enforcement of precise application-specific information security guarantees
It enables regression testing of information security guarantees.

PIDIN uses program dependence graphs to precisely and intuitively capture the information
flows within an entire program and a custom PDG query language to
allow the exploration, specification, and enforcement of information security guarantees. 
Developers can understand how information flows within a program and express precise security guarantees.
\section{GPUCC}
\label{sec:orgheadline3}

 GPUCC is an open-source gpgpu compiler. More precisely, an LLVM-based, fully open-source, 
CUDA compatible compiler for high performance computing.

The paper focuses on Clang frontend, IR generator and host code generator. 
 For CUDA frontend, gpucc has the capacity to  compile mixed code. gpucc increases 
the function inlining threshold to 1100, which we select based on the overall 
effects on our benchmarks. 


It was designed as a competitive substitute of NVIDIA nvcc. The results of the paper
 show that gpucc produces code that is faster than or on par with nvcc. 
It also produces some modern features in modern compilers.

\begin{figure}
\centering
\subfloat[{$\ N = 100$}]{
  \includegraphics[width=0.25 \textwidth]{p100}
}
\subfloat[{$\ N = 1000$}]{
  \includegraphics[width=0.25 \textwidth]{p1000}
}
\subfloat[$\ $Riferimento]{
  \includegraphics[width=0.25 \textwidth]{p}
} \\
\vspace{-0.4cm}
\subfloat[{$\ N = 100$}]{
  \includegraphics[width=0.25 \textwidth]{gj100}
} 
\subfloat[{$\ N = 1000$}]{
  \includegraphics[width=0.25 \textwidth]{gj1000}
} 
\subfloat[$\ $Riferimento]{
  \includegraphics[width=0.25 \textwidth]{gj}
} 
\vspace{-0.2cm}

\end{figure}



\section{PSLP: Padded SLP Automatic Vectorization}
\label{sec:orgheadline4}

Writing code that makes efficient use optimization methods like vectorizations and SIMDizations
is hard and leads to platform-specific implementations. PSLP is a vectorization algorithm that can vectorize
code containing non-isomorphic instruction sequences.

 It injects a small number of redundant instructions into
the code to transform non-isomorphic sequences into isomorphic
ones. The padded instruction sequence can then be successfully vectorized.

The experiments shows that PSLP can decrease the execution time by up to 63\%.


PSLP solves a major problem of existing SLP vectorization
algorithms, without having to worry about the instruction sets.

\section{Optimizing Large Programs at Small Price}
\label{sec:orgheadline5}

Optimization of a large program during compiling might cost a lot of time and memory. 
This paper presents the analysis of the causes of the difficulty for optimization on 
large scale programs. 

The common motive behind the unscalability problem is the large number of statements
 within a single static control part, which also amounts to large number of dependences. 


The proposed method for solving this problem is through Statement
 Condensation and Dependence Condensation algorithms. Where the program is then represented with just 
super-statements and representative dependences, which still allows to correctly
 reason about the application of any particular loop optimization. 

 In order to speed up the construc- tion of PDG while not sacrificing precision, 
this paper presented scalefuse, to this problem.
Scalefuse represents the entire hot region with a smaller set of 
statements (and dependences) that is sufficient for the compiler to perform loop optimizations. 
The proposed method however, might not be able to be very effective in solving the problem.

\section{HELIX}
\label{sec:orgheadline6}

HELIX is an automatic parallelization of irregular programs for chip multiprocessing. 
It was designed for loop parallelization that assigns successive iterations of a 
loop to separate threads.

HELIX applies code transformations to minimize the inefficiencies of sequential 
segments, data transfer, signaling, and thread management. It works on
 an intermediate representation of a program. Experimental results show that 
by reducing communication costs and increasing thread level parallelism,
 HELIX achieves a mean speedup of 2.25 (with a maximum of 4.12) 
over a broad range of applications on a real six-core system.

HELIX is fully automatic. It does not depend on source code annotations or 
modifications, or other forms of human intervention.

\section{Reactive Tiling}
\label{sec:orgheadline7}

This paper focusing on the discussion of manipulation and management of multicore system resources. 
In the real life application development, most of the effort were spent focusing on the OS and hardware 
problems. This paper shows a way for applications to react to OS and hardware management of the resources. 

For a typical matrix tiling problem. The proposed system will generate code pieces for different tile sizes. 
Then it  determines the best tile sizes for the cache allocations  using curve-fitting. Safe point analysis was introduced
in order to make sure the switching between different tile sizes are conducted seamlessly. 

Our experimental results show that the proposed scheme improves the performance of applications by 8.4\%, on average.

\section{Symbolic Range Analysis of Pointers}
\label{sec:orgheadline8}
Pointer analysis is one of the most fundamental techniques for compilers and stack based language
program designs. Mainstream compilers still struggle to distinguish intervals within the same array. 

This paper provides combination of pointer analysis with range analysis on the symbolic interval lattice including
global and local pointer disambiguation.  An abstract domain is introduced  in order to associates pointers with symbolic ranges. 
The symbolic analysis handles the subtleties of pointer arithmetic and memory indexation in a theoretical framework.

The analysis framework achieved precision in the proposed algorithm by combining alias analysis with classic range analysis on the symbolic domain. 
\end{document}
